\documentclass{article}

\usepackage[english]{babel}

% Set page size and margins
\usepackage[a4paper,top=2cm,bottom=2cm,left=3cm,right=3cm,marginparwidth=1.75cm]{geometry}

% Useful packages
\usepackage{amsmath}
\usepackage{graphicx}
\usepackage[colorlinks=true, allcolors=blue]{hyperref}

\title{\LARGE{\textbf{Algorithm Design 21/22}}\\ \vspace{1cm} Hands On 2 - Depth of a node in a random search tree}
\author{Federico Ramacciotti}
\date{}

\begin{document}
\maketitle

\section{Problem}
A random search tree for a set $S$ can be defined as follows: if $S$ is empty, then the null tree is a random search tree; otherwise, choose uniformly at random a key $k\in S$: the random search tree is obtained by picking $k$ as root, and the random search trees on $L = \{x \in S : x < k\}$ and $R = \{x \in S : x > k\}$ become, respectively, the left and right subtrees of the root $k$.
Consider the Randomized Quick Sort discussed in class and analyzed with indicator variables [CLRS 7.3], and observe that the random selection of the pivots follows the above process, thus producing a random search tree of $n$ nodes.
\begin{enumerate}
    \item Using a variation of the analysis with indicator variables $X_{ij}$, prove that the expected depth of a node (i.e. the random variable representing the distance of the node from the root) is nearly $2\log n$.
    \item Prove that the expected size of its subtree is nearly $2\log n$ too, observing that it is a simple variation of the previous analysis.
    \item Prove that the probability that the depth of a node exceeds $c*2*\log n$ is small for any given constant $c > 2$. [Note: it can be solved with Chernoff’s bounds as we know the expected value.]
\end{enumerate}
Note: $\sum_{k=1}^{n} \frac{1}{k} \leq \log n + 1$

\section{Solution}
Define an indicator variable 
$$X_{ij}=\begin{cases}
1 & \textrm{if}\ z_i\ \textrm{has been compared to}\ z_j\\
0 & \textrm{otherwise}
\end{cases}$$
Building the tree from a non-empty set $S$, every node is compared to all its ancestors. 

\subsection{Height}
The height of a node is the sum of all the indicator variables of its ancestors; since we cannot know that precisely, we can only observe that the height of a node $i$ is smaller than or equal to the sum of all the elements compared to $i$ (i.e. the sum of all the indicator variables $X_{ij}$, fixing $i$). So, given that the probability that $X_{ij}=1$ is $\frac{2}{j-i+1}$, the expected depth of a node is: 
\begin{align*}
E\left[d\left(i\right)\right]
&=E\left[\sum_{j\ \in\ ancestors(i)}X_{ij}\right]\leq E\left[\sum_{j=1}^{n}X_{ij}\right]\\
&=\sum_{j=1}^{n} E\left[X_{ij}\right]=\sum_{j=1}^{n} Pr\left[X_{ij}\right]\\
&=\sum_{j=1}^{n} \frac{2}{j-i+1}=2\sum_{j=1}^{n} \frac{1}{j-i+1}=O(2\log n)
\end{align*}

\newpage
\subsection{Size}
The expected size of a subtree rooted in a node $i$ is the number of all the descendants on $i$, since every one of them is compared to $i$ once. As observed in the previous analysis, it is again not possible to count only the descendants of a node. The solution is therefore an approximation, saying that the expected size of the subtree in $i$ is smaller than or equal to the sum of all the indicator variables for a fixed $i$ (i.e. the number of elements compared to $z_i$). $$E\left[size\left(i\right)\right]=E\left[\sum_{j\ \in\ descendants(i)}X_{ij}\right]\leq E\left[\sum_{j=1}^{n}X_{ij}\right]=O(2\log n)$$

\subsection{Depth}
Given the $X_{ij}$ indicator variables already defined, we define also $Y_i=\sum_{j\in ancestors(i)}X_{ij}$ as the depth of a node $i$. Using Chernoff's bounds with $\mu=E[Y_i]=2\log n$ and $\lambda = 2c\log n-2\log n = (2c-2)\log n$, we have:
\begin{align*}
Pr\left[Y_i > 2c\log n\right] &\leq e^{-\frac{((2c-2)\log n)^2}{2(2\log n)+(2c-2)\log n}}\\
&= e^{-\frac{(2c-2)^2(\log n)^2}{2c\log n + 2\log n}}\\
&= e^{-\frac{(2c-2)^2(\log n)^2}{(2c+2)\log n}}\\
&= e^{-\frac{(2c-2)^2\log n}{2c+2}}\\
&= n^{-\frac{2c^2-4c+2}{c+1}}\\
&= \frac{1}{n^{\frac{2c^2-4c+2}{c+1}}}
\end{align*}
This means that the probability that the depth of a node exceeds $2c\log n$ is small w.h.p. for any $c>2$. In fact, the exponent of $n$ is proportional to $c>2$ (i.e. the bigger $c$ the smaller the final result) and so the probability is small.

\end{document}