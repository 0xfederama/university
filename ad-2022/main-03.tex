\documentclass{article}

\usepackage[english]{babel}

% Set page size and margins
\usepackage[a4paper,top=2cm,bottom=2cm,left=3cm,right=3cm,marginparwidth=1.75cm]{geometry}

% Useful packages
\usepackage{amsmath}
\usepackage{graphicx}
\usepackage[colorlinks=true, allcolors=blue]{hyperref}
\usepackage{minted}

\title{\LARGE{\textbf{Algorithm Design 21/22}}\\ \vspace{1cm} Hands On 3 - Karp-Rabin Fingerprinting on strings}
\author{Federico Ramacciotti}
\date{}

\begin{document}
\maketitle

\section{Problem}
Given a string $S \equiv S[0\ldots n-1]$, and two positions $0 \leq i<j \leq n-1$, the longest common extension $lce(i, j)$ is the length of the maximal run of matching characters from those positions, namely: if $S[i] \neq S[j]$ then $lce(i, j) = 0$; otherwise, $lce(i, j) = max\{l \geq 1 : S[i\ldots i+l-1] = S[j\ldots j+l-1]\}$. For example, if $S = abracadabra$, then $lce(1, 2)=0$, $lce(0, 3)=1$, and $lce(0, 7)=4$.
Given $S$ in advance for pre-processing, build a data structure for $S$ based on the Karp-Rabin fingerprinting, in $O(n \log n)$ time, so that it supports subsequent online queries of the following two types:
\begin{itemize}
    \item $lce(i,j)$: it computes the longest common extension at positions $i$ and $j$ in $O(\log n)$ time.
    \item $equal(i,j,l)$: it checks if $S[i\ldots i+l-1]=S[j\ldots j+l-1]$ in constant time.
\end{itemize}

Analyze the cost and the error probability. The space occupied by the data structure can be $O(n \log n)$ but it is possible to use $O(n)$ space.

[Note: in this exercise, a one-time pre-processing is performed, and then many online queries are to be answered on the fly.]

\section{Solution}
A data structure can be built as follows: using prefix sums in an array $P$, in $P[i]$ store the hash value between $[0,i]$ using Karp-Rabin's rolling hash, so \mintinline{python}{P[i]=KR_hash(S[0...i])}. We can compute the hash value of $P[i,j]$ by doing $P[j]-P[i]$ and normalize it by dividing it by $\Sigma^i$ ($\Sigma=$ length of the alphabet). In order to compute $\Sigma^i$ in constant time we can use fast exponentiation, since we work modulo $p$ prime, and store all the values for any $0\leq i\leq n$ to have access to them in constant time.

The space used is $O(n)$ for both the $P$ and the $\Sigma$ arrays.

\subsection{Equal}
Define 
$$
equal(i,j,l)=\begin{cases}
\textrm{True} & \textrm{ if}\ (\frac{P[i+l-1]-P[i]}{\Sigma^i}=\frac{P[j+l-1]-P[j]}{\Sigma^i})\mod p\\
\textrm{False} & \textrm{ otherwise}
\end{cases}
$$
In order to divide by $\Sigma^i$ we can find the multiplicative inverse modulo $p$, since it is prime. It computes the solution in constant time and with $Pr[error]=\frac{1}{n^c}$, given by the probability that two different numbers have the same Karp-Rabin's hash value.

\subsection{Longest Common Extension}
$lce(i,j)$ uses exponential search in parallel on the portions of the arrays that start from $i$ and from $j$, comparing $\frac{P[i+k]-P[i]}{\Sigma^i}$ and $\frac{P[j+k]-P[j]}{\Sigma^i}$. As long as the hash values are equal (i.e. the substrings are equal w.h.p.), it searches to the right of that interval with $equal(i,j,k)$; when the two values are different, it searches in the latest interval found (e.g. in $k=16$ the two hashes are equal and at $k=32$ they're different: now the algorithm searches the interval $[16,32]$ with binary search, following the same method). The algorithm stops when the interval is one character long. 

This algorithm computes in $O(\log n)$ and, since it makes $\log n$ comparisons each with error $\frac{1}{n^c}$ (given by the probability that two different numbers have the same Karp-Rabin's hash value), the total error probability is $\frac{1}{n^c}\log n$.

\end{document}